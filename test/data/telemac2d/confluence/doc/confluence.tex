\chapter{Confluence}
%

% - Purpose & Problem description:
%     These first two parts give reader short details about the test case,
%     the physical phenomena involved and specify how the numerical solution will be validated
%
\section{Purpose}
%
To demonstrate that \telemac{2d} can model the flow that occurs at a river
confluence. The model represents the junction between two rectilinear laboratory channels
with rectangular cross-sections and constant slope.

%
\section{Description}

% - Geometry and Mesh:
%     This part describes the mesh used in the computation
%
%
\subsection{Geometry and Mesh}
%
The main channel is $10.8~m \times 0.8~m$ and the influent $3.2~m \times 0.5~m$.
The mesh is composed of $6168$ triangular elements and $3303$ nodes. 
Maximum size range: from $0.03$ to $0.1~m$.
The mesh is refined near the confluence as shown on Figure \ref{t2d:confluence:mesh}.

%
\begin{figure}[H]
  \centering
  \includegraphicsmaybe{[width=0.9\textwidth]}{../img/mesh.png}\\
  \includegraphicsmaybe{[width=0.9\textwidth]}{../img/mesh2.png}
  \caption{Mesh}\label{t2d:confluence:mesh}
\end{figure}

%
% - Initial and boundary conditions:
%     This part details both initial and boundary conditions used to simulate the case
%
%

\subsection{Bathymetry}

The main channel is $0.8~m$ broad whereas its influent is $0.5~m$ broad.
Both have a slope of $10^{-3}~m/m$.
The two channels join with an angle of $55$~\degree{}C.
The bathymetry is shown on figure \ref{t2d:confluence:bathy}.

\begin{figure}[H]
  \centering
  \includegraphicsmaybe{[width=\textwidth]}{../img/bathy.png}
  \caption{Bathymetry}\label{t2d:confluence:bathy}
\end{figure}

\subsection{Initial condition}

At initial time, there is no velocity in the channel and the free surface at rest is $0.2852~m$.

\subsection{Boundary Conditions}
%
Boundary conditions in the main channel are defined by:
\begin{enumerate}
\item[\textbullet] channel inlet: $Q = 0.07~m^3/s$
\item[\textbullet] channel outlet: $H = 0.2852~m$
\end{enumerate}

At the entrance of the influent channel a constant discharge of $Q = 0.035~m^3/s$ is imposed.
Lateral boundaries are solid walls with slip condition in the channel.

% - Physical parameters:
%     This part specifies the geometry, details all the physical parameters
%     used to describe both porous media (soil model in particularly) and
%     solute characteristics (dispersion/diffusion coefficients, soil <=> pollutant interactions...)
%
%
\subsection{Physical parameters}

At the bottom a Strickler formula with a friction coefficient of $62$ is set.
Trubulence is modeled by a constant viscosity equal to $10^{-3}~m^2/s$.

% - Numerical parameters:
%     This part is used to specify the numerical parameters used
%     (adaptive time step, mass-lumping when necessary...)
%
%
\subsection{Numerical parameters}
%
Duration of the simulation is set to $100s$ with a constant time step of $0.1~s$.
Numerical scheme parameters are the following:
\begin{enumerate}
  \item[\textendash] Type of advection:
  \begin{enumerate}
    \item[\textbullet] Characteristics on velocities (scheme n$^{\circ}1$)
    \item[\textbullet] Conservative + modified SUPG on depth (mandatory scheme)
  \end{enumerate}
  \item[\textendash] Type of element:
  \begin{enumerate}
    \item[\textbullet] Linear triangle (P1) for h and for velocities
  \end{enumerate}
  \item[\textendash] Solver : Conjugate gradient
  \item[\textendash] Solver accuracy: $10^{-4}$
  \item[\textendash] Implicitation for depth and for velocity: $1.0$
\end{enumerate}


%
% - Results:
%     We comment in this part the numerical results against the reference ones,
%     giving understanding keys and making assumptions when necessary.
%
%
\section{Results}

Initially the water level is horizontal.
In the main channel and in the lateral channel, the free surface increases with
time.
At the end of the calculation the water surface profile is constant in time
downstream and upstream from the confluence which shows that the computation has
converged.
The water depths in both channels (upstream and downstream) tend to be uniform.
The water level upstream from the confluence is $0.30~m$ higher than the water
level downstream.

\begin{figure}[H]
  \centering
   \subfloat[][free surface]{
  \includegraphicsmaybe{[width=0.9\textwidth]}{../img/waterDepth.png}}\\
  \subfloat[][velocity]{
  \includegraphicsmaybe{[width=0.9\textwidth]}{../img/velocity.png}}\\
  \subfloat[][velocity streamlines near the confluence]{
  \includegraphicsmaybe{[width=0.9\textwidth]}{../img/velocity_strm.png}}
  \caption{Results}\label{t2d:confluence:results}
\end{figure}

%
Close to the confluence, the water surface is rapidly varying.
The velocity field is regular in the whole domain (see Figure
\ref{t2d:confluence:results}).
No back eddy is computed at the junction of the two rivers with the turbulence
model used in this test case despite mesh refinement in this area; such back
eddy has been observed on physical model experiments (see ref.
\cite{KumarGurram1997}).

\section{Conclusions}
\telemac{2d} reproduces appropriately free surface variations at a river
confluence. However, in order to simulate in detail the flow pattern in such
conditions, more sophisticated turbulence model should be used (see e.g; test
case named ``cavity'').

%
%\section{Steering file}
%\begin{lstlisting}[language=TelemacCas]
%/---------------------------------------------------------------------
%/ TELEMAC2D Version v7p0
%/ Validation test case 15
%/---------------------------------------------------------------------
%/---------------------------------------------------------------------
%/ INPUT-OUTPUT, FILES
%/---------------------------------------------------------------------
%
%GEOMETRY FILE                   = geo_confluence.slf
%FORTRAN FILE                    = t2d_confluence.f
%BOUNDARY CONDITIONS FILE        = geo_confluence.cli
%RESULTS FILE                    = r2d_confluence.slf
%REFERENCE FILE                  = f2d_confluence.slf
%
%/---------------------------------------------------------------------
%/ INPUT-OUTPUT, GRAPHICS AND LISTING
%/---------------------------------------------------------------------
%
%LISTING PRINTOUT PERIOD         = 100
%VARIABLES FOR GRAPHIC PRINTOUTS = 'U,V,H,S,B'
%MASS-BALANCE                    = YES
%GRAPHIC PRINTOUT PERIOD         = 1000
%
%/---------------------------------------------------------------------
%/ PARAMETERS
%/---------------------------------------------------------------------
%
%FRICTION COEFFICIENT   = 62.
%LAW OF BOTTOM FRICTION = 3
%TURBULENCE MODEL       = 1
%VELOCITY DIFFUSIVITY   = 1.E-3
%
%/---------------------------------------------------------------------
%/ EQUATIONS, BOUNDARY CONDITIONS
%/---------------------------------------------------------------------
%
%VELOCITY PROFILES     = 2;2;2
%PRESCRIBED FLOWRATES  = 0.;0.035;0.070
%PRESCRIBED ELEVATIONS = 0.2852;0.;0.
%
%/---------------------------------------------------------------------
%/ EQUATIONS, INITIAL CONDITIONS
%/---------------------------------------------------------------------
%
%INITIAL ELEVATION  = 0.2852
%INITIAL CONDITIONS = 'CONSTANT ELEVATION'
%
%/---------------------------------------------------------------------
%/ INPUT-OUTPUT, INFORMATION
%/---------------------------------------------------------------------
%
%VALIDATION =YES
%TITLE      ='Validation test case 15'
%/---------------------------------------------------------------------
%/ NUMERICAL PARAMETERS
%/---------------------------------------------------------------------
%
%TIME STEP                       = 0.1
%NUMBER OF TIME STEPS            = 1000
%TREATMENT OF THE LINEAR SYSTEM  = 2
%TYPE OF ADVECTION               = 1;5
%SUPG OPTION                     = 2;2
%H CLIPPING                      = NO
%
%/---------------------------------------------------------------------
%/ NUMERICAL PARAMETERS, SOLVER
%/---------------------------------------------------------------------
%INFORMATION ABOUT SOLVER         = YES
%SOLVER                           = 1
%SOLVER OPTION                    = 3
%MASS-LUMPING ON H                = 1.
%IMPLICITATION FOR DEPTH          = 1.
%IMPLICITATION FOR VELOCITY       = 1.
%\end{lstlisting}

% Here is an example of how to include the graph generated by validateTELEMAC.py
% They should be in test_case/img
%\begin{figure} [!h]
%\centering
%\includegraphics[scale=0.3]{../img/mygraph.png}
% \caption{mycaption}\label{mylabel}
%\end{figure}


